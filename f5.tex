%%%%%%%%%%%%%%%%%%%%%%%%%%%%%%%%%%%%%%%%%%%%%%%%%%%%%%%%%%%%%%%%%%%%
%%%%%%%%%%%%%%%%%%%%%%%%%%%%%%%%%%%%%%%%%%%%%%%%%%%%%%%%%%%%%%%%%%%%
%
%DOCUMENT SETTINGS
%
%%%%%%%%%%%%%%%%%%%%%%%%%%%%%%%%%%%%%%%%%%%%%%%%%%%%%%%%%%%%%%%%%%%%
%%%%%%%%%%%%%%%%%%%%%%%%%%%%%%%%%%%%%%%%%%%%%%%%%%%%%%%%%%%%%%%%%%%%

\documentclass[12pt]{article}

\usepackage{amsmath,amssymb,amsthm,epsfig}

\evensidemargin 0in
\oddsidemargin 0in
\topmargin -.3in
\setlength{\textheight}{8.5in}
\setlength{\textwidth}{6.5in}

\newcommand{\ds}{\displaystyle}
\newcommand{\ul}{\underline}
\newcommand{\vs}{\vspace{3mm}}
\newcommand{\cF}{{\mathcal F}}

\newcounter{inner}

\newcommand{\op}[1]{\operatorname{#1}}
\newcommand{\bx}[1]{\makebox(8,5.5)[c]{#1}}

\newtheorem*{theorem}{Theorem}
\newtheorem*{claim}{Claim}
\begin{document}
%%%%%%%%%%%%%%%%%%%%%%%%%%%%%%%%%%%%%%%%%%%%%%%%%%%%%%%%%%%%%%%%%%%%
%%%%%%%%%%%%%%%%%%%%%%%%%%%%%%%%%%%%%%%%%%%%%%%%%%%%%%%%%%%%%%%%%%%%
%
%HEADER
%
%%%%%%%%%%%%%%%%%%%%%%%%%%%%%%%%%%%%%%%%%%%%%%%%%%%%%%%%%%%%%%%%%%%%
%%%%%%%%%%%%%%%%%%%%%%%%%%%%%%%%%%%%%%%%%%%%%%%%%%%%%%%%%%%%%%%%%%%%
\begin{center}
\hrule
\vskip .2in
\centerline{\bf \Large 21-127-T: Concepts of Mathematics}
\centerline{\bf Homework 5}
{\bf Due date: 6/19/2015, 11:59 PM}
\vskip .2in
\hrule
\end{center}
\thispagestyle{empty}
{\bf \noindent Name:Dina Yerlan \newline Collaborators:}
\vspace {0.2in}
\hrule
\vspace {0.2in}

%%%%%%%%%%%%%%%%%%%%%%%%%%%%%%%%%%%%%%%%%%%%%%%%%%%%%%%%%%%%%%%%%%%%
%%%%%%%%%%%%%%%%%%%%%%%%%%%%%%%%%%%%%%%%%%%%%%%%%%%%%%%%%%%%%%%%%%%%
%
%HOMEWORK BODY
%
%%%%%%%%%%%%%%%%%%%%%%%%%%%%%%%%%%%%%%%%%%%%%%%%%%%%%%%%%%%%%%%%%%%%
%%%%%%%%%%%%%%%%%%%%%%%%%%%%%%%%%%%%%%%%%%%%%%%%%%%%%%%%%%%%%%%%%%%%

%%%%%%%%%%%%%%%%%%%%%%%%%%%%%%%%%%%%%%%%%%%%%%%%%%%%%%%%%%%%%%%%%%%%
%Question #1
%%%%%%%%%%%%%%%%%%%%%%%%%%%%%%%%%%%%%%%%%%%%%%%%%%%%%%%%%%%%%%%%%%%%
\section*{Question 1}

Compute the following quantities:
%%%%%%%%%%%%%%%%%%%%%%%%%%%%%
%Question #1: Part a
%%%%%%%%%%%%%%%%%%%%%%%%%%%%%
\subsection*{Part a}

$(a+b)^{5}=(\text{mod }5)$\\

{\noindent\bf Answer:}
{
$(a+b)^{5}(\text{mod }5)=$\newline
$=a^{5}+b^{5}+5a^{4}b^{1}+10a^{3}b^{2}+10a^{2}b^{3}+5a^{1}b^{4}=$\newline
$=a^{5}+b^{5}+5(a^{4}b^{1}+2a^{3}b^{2}+2a^{2}b^{3}+a^{1}b^{4})(\text{mod }5)=$\newline
$= a^{5}+b^{5}$
}

%%%%%%%%%%%%%%%%%%%%%%%%%%%%%
%Question #1: Part b
%%%%%%%%%%%%%%%%%%%%%%%%%%%%%
\subsection*{Part b}

$(a+b)^{3}=? (\text{mod }3)$\\

{\noindent\bf Answer:}
{
$(a+b)^{3}(\text{mod }3)=$\newline
$=a^{3}+b^{3}+3a^{2}b+3ab^{2}=$\newline
$=a^{3}+b^{3}+3(a^{2}b+ab^{2})(\text{mod }3)=$\newline
$= a^{3}+b^{3}$
}

%%%%%%%%%%%%%%%%%%%%%%%%%%%%%
%Question #1: Part c
%%%%%%%%%%%%%%%%%%%%%%%%%%%%%
\subsection*{Part c}

$(a+b)^{2}=\text{? }(\text{mod }2)$\\

{\noindent\bf Answer:}

{
$(a+b)^{2}(\text{mod }2)=$\newline
$=a^{2}+b^{2}+2ab=$\newline
$=a^{2}+b^{2}+2ab (\text{mod }2)=$\newline
$= a^{2}+b^{2}$


}

%%%%%%%%%%%%%%%%%%%%%%%%%%%%%
%Question #1: Part d
%%%%%%%%%%%%%%%%%%%%%%%%%%%%%
\subsection*{Part d}

Let $p$ be a positive prime. $(a+b)^{p}=\text{? }(\text{mod }p)$\\

{\noindent\bf Answer:}
{
$(a+b)^{p}(\text{mod }p)=$\newline
Using binomial expansion:\newline
$(a+b)^{p}=\binom{p}{0}a^{p}b^{0}+\binom{p}{1}a^{p-1}b^{1}+...+\binom{p}{1}a^{1}b^{p-1}+\binom{p}{p}a^{0}b^{p}$\newline
 \newline
By computing binomial coefficients:\newline
$=a^{p}+b^{p}+p(a^{p-1}b^{1}+2a^{p-2}b^{2}+...2a^{2}b^{p-2}+a^{1}b^{p-1})=$\newline
$a^{p}+b^{p}+p(a^{p-1}b^{1}+2a^{p-2}b^{2}+...2a^{2}b^{p-2}+a^{1}b^{p-1})(\text{mod }p)=$\newline
$=a^{p}+b^{p}$\newline
So, $(a+b)^{p}(\text{mod }p)=a^{p}+b^{p}$\newline

}

\newpage
%%%%%%%%%%%%%%%%%%%%%%%%%%%%%%%%%%%%%%%%%%%%%%%%%%%%%%%%%%%%%%%%%%%%
%Question #2
%%%%%%%%%%%%%%%%%%%%%%%%%%%%%%%%%%%%%%%%%%%%%%%%%%%%%%%%%%%%%%%%%%%%
\section*{Question 2}

Let $p >1$ be an integer. Suppose the following property holds
Prove that p is prime.\\

{\noindent\bf Answer:}

{\noindent
Proof:\newline
By contradiction,assume $ \exists a,b$ s.t $ ab (mod) p=0 $ and $ a (mod) p \neq 0$ or $ b (mod)p \neq 0 $\newline
So, if $ ab(mod)p $ then $ \exists k \in Z$ s.t $ ab=pk $\newline
Then $a(mod)p \neq 0$ and $b(mod)p \neq 0$ \newline
Observe that this is absurd\newline
So there exists prime number $p$ that satisfies the above property

}
\newpage
%%%%%%%%%%%%%%%%%%%%%%%%%%%%%%%%%%%%%%%%%%%%%%%%%%%%%%%%%%%%%%%%%%%%
%Question #3
%%%%%%%%%%%%%%%%%%%%%%%%%%%%%%%%%%%%%%%%%%%%%%%%%%%%%%%%%%%%%%%%%%%%
\section*{Question 3}

%%%%%%%%%%%%%%%%%%%%%%%%%%%%%
%Question #3: Part a
%%%%%%%%%%%%%%%%%%%%%%%%%%%%%
\subsection*{Part a}

Let $\{S_{i}\}_{i\in\mathbb{N}}$ be a collection of subsets of the real line satisfying the following property
\begin{enumerate}
  \item $i<j\Rightarrow S_{j}\subseteq S_{i}$
  \item$\forall\epsilon>0,\exists n(\epsilon)\in\mathbb{N}$ s.t. $S_{n(\epsilon)}$ has measure at most $\epsilon$
\end{enumerate}
Prove that if $S\subseteq\bigcap_{i\in\mathbb{N}}S_{i}$ then S has measure zero.\\

{\noindent\bf Answer:}
{
 \newline
Proof:\newline
By definition of measure, if $\forall\epsilon>0,\exists n(\epsilon)\in\mathbb{N}$
s.t. $S_{n(\epsilon)}$ has measure at most $\epsilon$ then $\bigcap_{i\in\mathbb{N}}S_{i}$ has measure zero.\newline
Choose  $S_{i}\subseteq\bigcap_{i\in\mathbb{N}}S_{i}$\newline
Then $\forall\epsilon>0,\exists n(\epsilon)\in\mathbb{N}$
$S_{i}$ has $\epsilon>0$\newline
Observe that subset has the same sequence of intervals\newline
Then $S$ has measure zero.\newline
}
%%%%%%%%%%%%%%%%%%%%%%%%%%%%%
%Question #3: Part b
%%%%%%%%%%%%%%%%%%%%%%%%%%%%%
\subsection*{Part b}

Prove the Cantor Set has measure zero.\\

{\noindent\bf Answer:}
{
 \newline
Proof:Let $C_{k}$ denote the $kth$ element of Cantor set C\newline
According to Cantor set definition, $C_{0}=[0,1]$,$[0,1/3] \cup [2/3,1]$ and so on,
and $\bigcap_{i}C_{i}$\newline
Adding up size of intervals of Cantor set gives us:\newline
$1/3+2(1/3)^{2}+2^{2}(1/3)^{2}+2^{3}(1/3)^{3}+....+2^{i-1}(1/3)^{i}$\newline
So C is the complement of a set of size $1/3+2(1/3)^{2}+2^{2}(1/3)^{2}+2^{3}(1/3)^{3}+....
=1$,observe that it has measure 0 as it has 1:1 correspondence with rational numbers\newline

}
\newpage
%%%%%%%%%%%%%%%%%%%%%%%%%%%%%%%%%%%%%%%%%%%%%%%%%%%%%%%%%%%%%%%%%%%%
%Question #4
%%%%%%%%%%%%%%%%%%%%%%%%%%%%%%%%%%%%%%%%%%%%%%%%%%%%%%%%%%%%%%%%%%%%
\section*{Question 4}

Do problem 16 from page 105.\\

{\noindent\bf Answer:}
{
 \newline
Proof:\newline
By definition of upper bound, $\exists x \in R$ s.t $x \geq s,\forall s \in S$\newline
Observe that the complement of upper bound is $x \in R$ s.t $x<s, \forall s \in S$\newline
We see that the complement of $U(S)$ is an open set\newline
The set is closed if its complement is open\newline
Then by definition, $U(S)$ is closed set\newline

}
\newpage
%%%%%%%%%%%%%%%%%%%%%%%%%%%%%%%%%%%%%%%%%%%%%%%%%%%%%%%%%%%%%%%%%%%%
%Question #5
%%%%%%%%%%%%%%%%%%%%%%%%%%%%%%%%%%%%%%%%%%%%%%%%%%%%%%%%%%%%%%%%%%%%
\section*{Question 5}

Formulate modular equality as a relation and show it's reflexive, symmetric, and transitive.\\

{\noindent\bf Answer:}
{
  \newline
Reflexive.Equality is reflexive since for each $x \in R$, $x=x$\newline
 \newline
Symmetric. Equality is symmetric since for each $x,y \in R$ if $x=y$ then $y=x$\newline
 \newline
Transitive. Equqlity is transitive since for each $x,y,z \in R$ if $x=y$ and $y=z$ then $x=z$\newline
}

\end{document}
