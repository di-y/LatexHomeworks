%%%%%%%%%%%%%%%%%%%%%%%%%%%%%%%%%%%%%%%%%%%%%%%%%%%%%%%%%%%%%%%%%%%%
%%%%%%%%%%%%%%%%%%%%%%%%%%%%%%%%%%%%%%%%%%%%%%%%%%%%%%%%%%%%%%%%%%%%
%
%DOCUMENT SETTINGS
%
%%%%%%%%%%%%%%%%%%%%%%%%%%%%%%%%%%%%%%%%%%%%%%%%%%%%%%%%%%%%%%%%%%%%
%%%%%%%%%%%%%%%%%%%%%%%%%%%%%%%%%%%%%%%%%%%%%%%%%%%%%%%%%%%%%%%%%%%%

\documentclass[12pt]{article}

\usepackage{amsmath,amssymb,amsthm,epsfig}

\evensidemargin 0in
\oddsidemargin 0in
\topmargin -.3in
\setlength{\textheight}{8.5in}
\setlength{\textwidth}{6.5in}

\newcommand{\ds}{\displaystyle}
\newcommand{\ul}{\underline}
\newcommand{\vs}{\vspace{3mm}}
\newcommand{\cF}{{\mathcal F}}

\newcounter{inner}

\newcommand{\op}[1]{\operatorname{#1}}
\newcommand{\bx}[1]{\makebox(8,5.5)[c]{#1}}

\newtheorem*{theorem}{Theorem}
\newtheorem*{claim}{Claim}
\begin{document}
%%%%%%%%%%%%%%%%%%%%%%%%%%%%%%%%%%%%%%%%%%%%%%%%%%%%%%%%%%%%%%%%%%%%
%%%%%%%%%%%%%%%%%%%%%%%%%%%%%%%%%%%%%%%%%%%%%%%%%%%%%%%%%%%%%%%%%%%%
%
%HEADER
%
%%%%%%%%%%%%%%%%%%%%%%%%%%%%%%%%%%%%%%%%%%%%%%%%%%%%%%%%%%%%%%%%%%%%
%%%%%%%%%%%%%%%%%%%%%%%%%%%%%%%%%%%%%%%%%%%%%%%%%%%%%%%%%%%%%%%%%%%%
\begin{center}
\hrule
\vskip .2in
\centerline{\bf \Large 21-127-T: Concepts of Mathematics}
\centerline{\bf Homework 2}
{\bf Due date: 5/30/2015, 11:00 PM}
\vskip .2in
\hrule
\end{center}
\thispagestyle{empty}
{\bf \noindent Name:Dina Yerlan \newline Collaborators:0}
\vspace {0.2in}
\hrule
\vspace {0.2in}

%%%%%%%%%%%%%%%%%%%%%%%%%%%%%%%%%%%%%%%%%%%%%%%%%%%%%%%%%%%%%%%%%%%%
%%%%%%%%%%%%%%%%%%%%%%%%%%%%%%%%%%%%%%%%%%%%%%%%%%%%%%%%%%%%%%%%%%%%
%
%HOMEWORK BODY
%
%%%%%%%%%%%%%%%%%%%%%%%%%%%%%%%%%%%%%%%%%%%%%%%%%%%%%%%%%%%%%%%%%%%%
%%%%%%%%%%%%%%%%%%%%%%%%%%%%%%%%%%%%%%%%%%%%%%%%%%%%%%%%%%%%%%%%%%%%

%%%%%%%%%%%%%%%%%%%%%%%%%%%%%%%%%%%%%%%%%%%%%%%%%%%%%%%%%%%%%%%%%%%%
%Question #1
%%%%%%%%%%%%%%%%%%%%%%%%%%%%%%%%%%%%%%%%%%%%%%%%%%%%%%%%%%%%%%%%%%%%
\section*{Question 1}

Do problem 4, parts (a)-(c) from section 1.6. Note that the statement $v$ is easier to understand if written as
\[
v\equiv\{\exists M,K\in\mathbb{R}\text{ s.t. }\forall x\in\mathbb{R}, f(x)>M\}
\]
%%%%%%%%%%%%%%%%%%%%%%%%%%%%%
%Question #1: Part a
%%%%%%%%%%%%%%%%%%%%%%%%%%%%%
\subsection*{Part a - Answer:}
{
$p$ $\equiv$ $\{\forall M \in R \exists x \in D \text{ s.t } f(x)>M\}$\newline
 \newline
$v$ $\equiv$ $\{\exists M,K \in R \text{ s.t } \forall x \in R,f(x)>M\}$\newline
 \newline
We need to prove or disprove: $p$ $\Rightarrow$ $v$\newline
Assume it is false\newline
Counterexample: Let $f(x)= 1/x$, it satisfies $p$\newline
because $\{\forall M \in R \exists x \in D \text{ s.t } 1/x>M\}$\newline
Then let $M_{0}=2$ $K_{0}=0$ and $x_{0}=1$\newline
We see that this does not satisfy $v$
}
%%%%%%%%%%%%%%%%%%%%%%%%%%%%%
%Question #1: Part b
%%%%%%%%%%%%%%%%%%%%%%%%%%%%%
\subsection*{Part b - Answer:}
{
$p$ $\equiv$ $\{\forall M \in R \exists x \in D \text{st} f(x)>M\}$\newline
 \newline
$v$ $\equiv$ $\{\exists M,K \in R \text{st} \forall x \in R,f(x)>M\}$\newline
 \newline
We need to prove or disprove: $v$ $\Rightarrow$ $p$\newline
Answer: $v$ $\Rightarrow$ $p$ is false\newline
Counterexample: Let $f(x)= 1$, it satisfies $v$\newline
because $\{\exists M,K \in R \text{st} \forall x \in R,1>M\}$\newline
Then let $M_{0}=-1$ $K_{0}=0$ and $x_{0}=1$\newline
We see that this does not satisfy $p$
}
%%%%%%%%%%%%%%%%%%%%%%%%%%%%%
%Question #1: Part c
%%%%%%%%%%%%%%%%%%%%%%%%%%%%%
\subsection*{Part c - Answer:}
{
$q$ $\equiv$ $\{\forall M,K \in R \forall x>K,f(x)>M\}$\newline
 \newline
$v$ $\equiv$ $\{\exists M,K \in R \text{st} \forall x \in R,f(x)>M\}$\newline
We need to prove or disprove: $q$ $\Rightarrow$ $v$\newline
Assume $q$ is true\newline
To prove v, we must choose $x_{0}$ we must find $M_{0}$ s.t $f(x_{0})>M_{0}$\newline
Since q is true, we know that there exists $x_{0}$ s.t $f(x_{0})>M_{0}$\newline
Let $M=M_{0}-1$ and we know $f(x_{0})>M_{0}$\newline
there exists such $x_{0}$ and $M_{0}$\newline
s.t $f(x_{0})>M_{0}-1$\newline
$f(x_{0})>M$
So $v$ is also true $\square$
}
\newpage
%%%%%%%%%%%%%%%%%%%%%%%%%%%%%%%%%%%%%%%%%%%%%%%%%%%%%%%%%%%%%%%%%%%%
%Question #2
%%%%%%%%%%%%%%%%%%%%%%%%%%%%%%%%%%%%%%%%%%%%%%%%%%%%%%%%%%%%%%%%%%%%
\section*{Question 2}

Do problem 2, parts (d)-(f) from section 1.6. Note that the statement $v$ is easier to understand if written as
\[
v\equiv\{\exists M,K\in\mathbb{R}\text{ s.t. }\forall x\in\mathbb{R}, f(x)>M\}
\]
%%%%%%%%%%%%%%%%%%%%%%%%%%%%%
%Question #2: Part d
%%%%%%%%%%%%%%%%%%%%%%%%%%%%%
\subsection*{Part d - Answer:}
{
$q$ $\equiv$ $\{\exists M \in R \text{st} \forall x \in D, f(x)>M\}$\newline
 \newline
$s$ $\equiv$ $\{\forall M,K \in R, \forall x>K,f(x)>M\}$\newline
 \newline
We need to prove or disprove: $s$ $\Rightarrow$ $q$\newline
Answer: $s$ $\Rightarrow$ $q$ is false\newline
Counterexample: Let $f(x)=x^{2}$, it satisfies $s$\newline
because $\{\forall M,K \in R, \forall x>K,x^{2}>M\}$\newline
Then let $M_{0}=2$ $K_{0}=-100$ and $x_{0}=0$\newline
So, $0>2$ and it is absurd\newline
We see that $q$ does not satisfy these values\newline
}
%%%%%%%%%%%%%%%%%%%%%%%%%%%%%
%Question #2: Part e
%%%%%%%%%%%%%%%%%%%%%%%%%%%%%
\subsection*{Part e - Answer:}
{
$r$ $\equiv$ $\{\exists M,K \in R s.t \exists x>K, f(x)>M\}$\newline
 \newline
$s$ $\equiv$ $\{\forall M,K \in R, \forall x>K,f(x)>M\}$\newline
 \newline
We need to prove or disprove: $r$ $\Rightarrow$ $s$\newline
Answer: $r$ $\Rightarrow$ $s$ is false\newline
Counterexample: Let $f(x)= 1$, it satisfies $r$\newline
because $\{\exists M,K \in R s.t \exists x>K, 1>M\}$\newline
Then let $M_{0}=-1$ $K_{0}=0$ and $x_{0}=0$\newline
We see that this does not satisfy $s$
}
%%%%%%%%%%%%%%%%%%%%%%%%%%%%%
%Question #2: Part f
%%%%%%%%%%%%%%%%%%%%%%%%%%%%%
\subsection*{Part f - Answer:}
{
$r$ $\equiv$ $\{\exists M,K \in R s.t \exists x>K, f(x)>M\}$\newline
 \newline
$s$ $\equiv$ $\{\forall M,K \in R, \forall x>K,f(x)>M\}$\newline
We need to prove or disprove: $s$ $\Rightarrow$ $r$\newline
Answer: $s$ $\Rightarrow$ $r$ is false\newline
Counterexample: Let $f(x)= sin(x)$, it satisfies $s$\newline
as $\{\forall M,K \in R, \forall x>K,sin(x)>M\}$
Then let $M_{0}=2$ $K_{0}=0$ and $x_{0}=0$\newline
We see that this does not satisfy $r$
}

\newpage
%%%%%%%%%%%%%%%%%%%%%%%%%%%%%%%%%%%%%%%%%%%%%%%%%%%%%%%%%%%%%%%%%%%%
%Question #3
%%%%%%%%%%%%%%%%%%%%%%%%%%%%%%%%%%%%%%%%%%%%%%%%%%%%%%%%%%%%%%%%%%%%
\section*{Question 3}

Prove the following claim:
\begin{claim}
There exists no largest real number.
\end{claim}

{\noindent\bf Answer:}
{
 \newline
 \newline
Claim: There exists no largest real number\newline
 \newline
Given(negation of the claim):  $\{\exists M\in R\text{ s.t. } \forall x\in R\ M>x \}$\newline
 \newline
Proof: By contradiction assume there exists the largest real number\newline
Choose $M_{0}\in R$ s.t $\forall x\in R$ $M_{0}>x$\newline
Let $x=M_{0}+1$\newline
Then, $M_{0}>M_{0}+1$\newline
Witness that $M_{0}>M_{0}+1$\newline
$M_{0}>M_{0}+1$ the number is larger than number plus one\newline
This is absurd\newline
Thus our assumption was wrong$\square$\newline
}
\newpage
%%%%%%%%%%%%%%%%%%%%%%%%%%%%%%%%%%%%%%%%%%%%%%%%%%%%%%%%%%%%%%%%%%%%
%Question #4
%%%%%%%%%%%%%%%%%%%%%%%%%%%%%%%%%%%%%%%%%%%%%%%%%%%%%%%%%%%%%%%%%%%%
\section*{Question 4}

Suppose $D$ is some set and $P(x)$ is a predicate function. Mathematicians will often write claims of the form:\\

{\it\noindent There exists a unique (meaning there's one and ONLY one) element of a given set, $D$, satisfying a given property, $P(x)$.}
%%%%%%%%%%%%%%%%%%%%%%%%%%%%%
%Question #4: Part a
%%%%%%%%%%%%%%%%%%%%%%%%%%%%%
\subsection*{Part a}
Let $D$ be a set, $P(x)$ be a predicate function, and $x_{0}\in D$ s.t. $P(x_{0})$ is true. Write what it means for $x_{0}$ to be the unique element of $D$ satisfying $P(x)$. Also, write the negation of this statement. (HINT: The non-negated statement will involve the quantifier $\forall$).\newline
 \newline
{\noindent\bf Answer:}
{
 \newline
Claim: $\{\exists ! x_{0}\in D\text{ s.t. }P(x_{0})\}$\newline
\newline
Meaning: It means that there is one and only one element in D,$x_{0}$ that makes the expression $P(x_{0})$ true\newline
Negation: $\forall  x_{0}(\neg P(x_{0}) \lor \exists (P(y) \land y \neq x_{0}))$\newline
}
%%%%%%%%%%%%%%%%%%%%%%%%%%%%%
%Question #4: Part b
%%%%%%%%%%%%%%%%%%%%%%%%%%%%%
\subsection*{Part b}
Using your answer above, write the statement $\{\exists ! x\in D\text{ s.t. }P(x)\}$ using the quantifiers $\forall$ and $\exists$.\newline
{\noindent\bf Answer:}

{$(\exists x,P(x)) \land (\forall y,P(y))$ $\Rightarrow$ $(x=y)$
 \newline
 \newline
 \newline
 \newline
 \newline
 \newline
 \newline
 \newline
 \newline
 \newline
 \newline
 \newline
}

%%%%%%%%%%%%%%%%%%%%%%%%%%%%%
%Question #4: Part c
%%%%%%%%%%%%%%%%%%%%%%%%%%%%%
\subsection*{Part c}

Prove the following claim
\begin{claim}
If n is an even integer there exists a unique $i\in\mathbb{Z}$ s.t. $n=2i$.
\end{claim}

{\noindent\bf Answer:}
{
Claim:$\{n \in Z,\exists ! i\in Z\text{ s.t. }n=2i \}$\newline
Proof(Step 1): Choose $n_{0} \in Z$ s.t $n_{0}=2i_{0}$\newline
Let $i_{0}=1,n_{0}=2$\newline
We see that $2=2*1$, so $n_{0}=2i_{0}$ is true\newline
Proof(Step2):\newline
We need to prove that $i_{0}$ is unique\newline
By contradiction, assume  $i_{0}$ is not unique\newline
Choose $i_{0}\in Z$ s.t  $n_{0}=2(i_{0}+1)$ is also true\newline
Then, we are given that $n_{0}=2(i_{0}+1)$ and $n_{0}=2i_{0}$ should both be true\newline
We see that $2(i_{0}+1)=2i_{0}$ $=$\newline
$=$ $2i_{0}+2=2i_{0}$\newline
Observe that $2i_{0}+2=2i_{0}$ cannot be true\newline
Thus we ran into contradiction.$\square$

}
\newpage
%%%%%%%%%%%%%%%%%%%%%%%%%%%%%%%%%%%%%%%%%%%%%%%%%%%%%%%%%%%%%%%%%%%%
%Question #5
%%%%%%%%%%%%%%%%%%%%%%%%%%%%%%%%%%%%%%%%%%%%%%%%%%%%%%%%%%%%%%%%%%%%
\section*{Question 5}

Prove the following claim by contraposition:
\begin{claim}
Let $n,m,p\in\mathbb{Z}$. If $nmp$ is even then $n$, $m$, or $p$ is even.
\end{claim}

{\noindent\bf Answer:}
{
Contraposition: $n,m,p\in\mathbb{Z}$  $\{n \in O \land m \in O \land p \in O\}$ $\Rightarrow$  $\{nmp\in O\}$ \newline
Proof: By contraposition. Choose $n_{0},m_{0},p_{0} \in Z $ \newline
s.t $n_{0},m_{0},p_{0} \in O $\newline
then $\exists a,b,c \in Z$ where $n_{0}=2a+1$, $m_{0}=2b+1$, $p_{0}=2c+1$\newline
Observe that  $n_{0}m_{0}p_{0}$=$(2a+1)(2b+1)(2c+1)$ \newline
              $=$ $(4ab+2a+2b+1)(2b+1) $ $=$\newline
              $=$ $8ab^{2}+8ab+4b^{2}+4b+2a+1$ $=$\newline
              $=$ $2(4ab^{2}+4ab+2b^{2}+2b+a)+1$\newline
Let $k=4ab^{2}+4ab+2b^{2}+2b+a $ then $n_{0}m_{0}p_{0}=2k+1$\newline
We see that $n_{0}m_{0}p_{0}$ is odd, so we proved that $\{n_{0}m_{0}p_{0}\in O\}$ $\square$\newline
}
\end{document}
