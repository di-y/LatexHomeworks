%%%%%%%%%%%%%%%%%%%%%%%%%%%%%%%%%%%%%%%%%%%%%%%%%%%%%%%%%%%%%%%%%%%%
%%%%%%%%%%%%%%%%%%%%%%%%%%%%%%%%%%%%%%%%%%%%%%%%%%%%%%%%%%%%%%%%%%%%
%
%DOCUMENT SETTINGS
%
%%%%%%%%%%%%%%%%%%%%%%%%%%%%%%%%%%%%%%%%%%%%%%%%%%%%%%%%%%%%%%%%%%%%
%%%%%%%%%%%%%%%%%%%%%%%%%%%%%%%%%%%%%%%%%%%%%%%%%%%%%%%%%%%%%%%%%%%%

\documentclass[12pt]{article}

\usepackage{amsmath,amssymb,amsthm,epsfig}

\evensidemargin 0in
\oddsidemargin 0in
\topmargin -.3in
\setlength{\textheight}{8.5in}
\setlength{\textwidth}{6.5in}

\newcommand{\ds}{\displaystyle}
\newcommand{\ul}{\underline}
\newcommand{\vs}{\vspace{3mm}}
\newcommand{\cF}{{\mathcal F}}

\newcounter{inner}

\newcommand{\op}[1]{\operatorname{#1}}
\newcommand{\bx}[1]{\makebox(8,5.5)[c]{#1}}

\newtheorem*{theorem}{Theorem}
\newtheorem*{claim}{Claim}
\begin{document}
%%%%%%%%%%%%%%%%%%%%%%%%%%%%%%%%%%%%%%%%%%%%%%%%%%%%%%%%%%%%%%%%%%%%
%%%%%%%%%%%%%%%%%%%%%%%%%%%%%%%%%%%%%%%%%%%%%%%%%%%%%%%%%%%%%%%%%%%%
%
%HEADER
%
%%%%%%%%%%%%%%%%%%%%%%%%%%%%%%%%%%%%%%%%%%%%%%%%%%%%%%%%%%%%%%%%%%%%
%%%%%%%%%%%%%%%%%%%%%%%%%%%%%%%%%%%%%%%%%%%%%%%%%%%%%%%%%%%%%%%%%%%%
\begin{center}
\hrule
\vskip .2in
\centerline{\bf \Large 21-127-T: Concepts of Mathematics}
\centerline{\bf Homework 4}
{\bf Due date: 6/12/2015, 11:59 PM}
\vskip .2in
\hrule
\end{center}
\thispagestyle{empty}
{\bf \noindent Name:Dina Yerlan \newline Collaborators:}
\vspace {0.2in}
\hrule
\vspace {0.2in}

%%%%%%%%%%%%%%%%%%%%%%%%%%%%%%%%%%%%%%%%%%%%%%%%%%%%%%%%%%%%%%%%%%%%
%%%%%%%%%%%%%%%%%%%%%%%%%%%%%%%%%%%%%%%%%%%%%%%%%%%%%%%%%%%%%%%%%%%%
%
%HOMEWORK BODY
%
%%%%%%%%%%%%%%%%%%%%%%%%%%%%%%%%%%%%%%%%%%%%%%%%%%%%%%%%%%%%%%%%%%%%
%%%%%%%%%%%%%%%%%%%%%%%%%%%%%%%%%%%%%%%%%%%%%%%%%%%%%%%%%%%%%%%%%%%%

%%%%%%%%%%%%%%%%%%%%%%%%%%%%%%%%%%%%%%%%%%%%%%%%%%%%%%%%%%%%%%%%%%%%
%Question #1
%%%%%%%%%%%%%%%%%%%%%%%%%%%%%%%%%%%%%%%%%%%%%%%%%%%%%%%%%%%%%%%%%%%%
\section*{Question 1}

%%%%%%%%%%%%%%%%%%%%%%%%%%%%%
%Question #1: Part a
%%%%%%%%%%%%%%%%%%%%%%%%%%%%%
\subsection*{Part a}

Prove that a composition of bijections is bijective.\\

{\noindent\bf Answer:}
{
Proof: Let $f:Y \rightarrow Z$and $g:Z  \rightarrow Y$ be our bijective functions\newline
To prove $f \circ g$ is bijective, we need to prove that it is both injective and surjective.\newline
First,let's prove that $f\circ g$ is injective\newline
Choose $x_{0}$,$y_{0}$,$c \in Y$  s.t  $f\circ g(x_{0})= g\circ f(y_{0})=c$\newline
This means that $g(f(x))=g(f(y))$\newline
Let $f(x)=a,f(y)=b$ so $g(a)=g(b)$\newline
Since $g: Z  \rightarrow Y$ is injective and $g(a)=g(b)$, we know that $a=b$\newline
This means that $f(x_{0})=f(y_{0})$\newline
Since $f:Y  \rightarrow Z$ is injective and $f(x_{0})=f(y_{0})$, we know that $x_{0}=y_{0}$\newline
Thus we have shown that if $f \circ g(x)= g \circ f(y)$, $x=y$\newline
So, $f\circ g$ is injective.\newline
 \newline
Second, we need to prove $f\circ g$ is surjective\newline
Choose $y  \in Y$ \newline
Since $g: Z-Y$ is surjective, $\exists z \in Z $ s.t $g(z)=y$\newline
Since $f:Y-Z$ is surjective,$\exists x \in X $ s.t $f(x)=z$\newline
So, $ g \circ f(x)= g(f(x))=g(z)=y$\newline
Thus $f\circ g$ is surjective\newline
We proved that  $f\circ g$ is injective and surjective, so it is bijective $\square$\newline
 \newline
 \newline
 \newline

}
%%%%%%%%%%%%%%%%%%%%%%%%%%%%%
%Question #1: Part bPPPPP
%%%%%%%%%%%%%%%%%%%%%%%%%%%%%
\subsection*{Part b}

Prove that a function has an inverse iff it's bijective.\\

{\noindent\bf Answer:}
{
 \newline
Proof:
Let function be $f:A \rightarrow B$\newline
First, we need to prove if f has inverse then it is bijection\newline
Suppose $f$ has an inverse $g=f^-1$\newline
$\forall x \in B$ $\exists y$ s.t $y=f(x)$\newline
Now $f(y)=f(g(x))=x$ so f is surjective by definition of surjection\newline
$f(a)=f(b)$, applying $g$ on both sides we have $g(f(a))=g(f(b))$ so $a=b$\newline
So $f$ is injective by definition of injection\newline
$f$ is both injective and surjective so it must be bijective\newline
 \newline
Second, we need to prove that if  $f$ is a bijection then it has an inverse\newline
Suppose $f$ is bijective\newline
By definition, given $x \in B$ and $y \in A$ with $f(y)=x$\newline
Since $f$ is injective, there is only one $y$ s.t $g(x)=y$\newline
Now $\forall y, g(f(y))=y$ and  $\forall x, g(f(x))=x$ so $f^{-1}=g$ by arithmetic\newline
Thus if $f$ is bijective then it has inverse\newline
Conclusion: We proved that  if f has inverse then it is bijection and if $f$ is bijective then it has inverse.$\square$\newline

}
%%%%%%%%%%%%%%%%%%%%%%%%%%%%%
%Question #1: Part c
%%%%%%%%%%%%%%%%%%%%%%%%%%%%%
\subsection*{Part c}

Let $R,S,T$ be sets. We say that $\vert S\vert=\vert T\vert$ if there's a bijection between S and T. We remarked that ``='' works a bit differently for sets; namely, we must prove facts like $\vert S\vert=\vert T\vert$ implies $\vert T\vert=\vert S\vert$. You'll prove more. Prove that:
\begin{itemize}
  \item (Reflexivity): $\vert S\vert=\vert S\vert$
    \item (Symmetry): $\vert S\vert=\vert T\vert$ implies $\vert T\vert=\vert S\vert$
      \item (Transitivity): $\vert S\vert=\vert T\vert$ and $\vert T\vert=\vert R\vert$ implies $\vert S\vert=\vert R\vert$
\end{itemize}
We say that ``='' is an equivalence relation. We'll cover these in greater detail later. (HINT: Use the previous two parts.)\\

{\noindent\bf Answer:}
{
\newline
Proof(Reflexivity): We want to show that $\vert S\vert=\vert S\vert$\newline
Let $f:S \rightarrow S$ be indentity function, that is,$\forall s \in S$ $f(s)=s$\newline
This is a bijection since $\forall a,b \in S$,$f(a)=f(b)$ iff $a=b$\newline
Thus $f$ is a bijection of$ S$ onto $S$. We have $\vert S\vert=\vert S\vert$\newline
 \newline
Proof(Symmetry):
$\vert S\vert=\vert T\vert$, by definition there exists bijections $f:S\rightarrowT$\newline
Since $f$ is bijection, it has an inverse function $g:T\rightarrowS$ which is also bijection\newline
This shows that $\vert T\vert=\vert S\vert$\newline
 \newline
Proof(Transitivity):\newline
$\vert S\vert=\vert T\vert$ and $\vert T\vert=\vert R\vert$\newline
By definition, there are bijections $f:S\rightarrowT$ $f:T\rightarrowR$\newline
Let $h$ be a composition of $f$ and $g$\newline
Then there is bijection $h:S\rightarrowR$\newline
Then by reflexivity and symmetry $\vert S\vert=\vert R\vert$
}

\newpage
%%%%%%%%%%%%%%%%%%%%%%%%%%%%%%%%%%%%%%%%%%%%%%%%%%%%%%%%%%%%%%%%%%%%
%Question #2
%%%%%%%%%%%%%%%%%%%%%%%%%%%%%%%%%%%%%%%%%%%%%%%%%%%%%%%%%%%%%%%%%%%%
\section*{Question 2}

Do problem 6 on page 94.\\

{\noindent\bf Answer:}
{
 \newline
Let $f:Z \rightarrow Z^+$ defined by $f(x)=\vert x\vert$\newline
We see that function takes any integer and produces positive integers\newline
The function is one-to-one:\newline
Suppose $\forall x,y \in Z$ s.t $f(x)=f(y)$ \newline
Then we have $x=x$, and hence $x=y$\newline
The function is onto:\newline
Given any $x \in Z$ is also in $Z$ and $f(x-0)=(x-0)+0=x$\newline
Thus the function is injective and surjective $\square$
}

\newpage
%%%%%%%%%%%%%%%%%%%%%%%%%%%%%%%%%%%%%%%%%%%%%%%%%%%%%%%%%%%%%%%%%%%%
%Question #3
%%%%%%%%%%%%%%%%%%%%%%%%%%%%%%%%%%%%%%%%%%%%%%%%%%%%%%%%%%%%%%%%%%%%
\section*{Question 3}

Let $n,m,k\in\mathbb{N}$. Prove the following claims:
%%%%%%%%%%%%%%%%%%%%%%%%%%%%%
%Question #3: Part a
%%%%%%%%%%%%%%%%%%%%%%%%%%%%%
\subsection*{Part a}

$2^{n}=\sum_{i=0}^{n}\binom{n}{i}$\\

{\noindent\bf Answer:}
{
 \newline
Base Case:$\binom{0}{0}=1$ holds by definition.\newline
\newline
Induction Hypothesis: $\sum_{i=0}^{k}\binom{k}{i}=2^{k}$\newline
If $P(k)$ is true , where $k \geq 1$, then $P(k+1)$ is true.\newline
 \newline
Induction Step:
By strong induction, choose $k$ s.t $\sum_{i=0}^{k}\binom{k}{i}=2^{k}$ is true\newline
 $\sum_{i=0}^{k+1}\binom{k+1}{i}=\binom{k+1}{0}+\sum_{i=1}^{k}\binom{k+1}{i}+\binom{k+1}{k+1}$\newline
$= \binom{k+1}{0}+\sum_{i=1}^{k}\binom{k+1}{i-1} + \sum_{i=1}^{k}\binom{k}{i}+\binom{k+1}{k+1}$\newline
$=\sum_{i=0}^{k-1}\binom{k}{i}+\sum_{i=0}^{k-1}\binom{k+1}{k+1}+ \binom{k+1}{0}+\sum_{i=1}^{k}\binom{k}{i}$\newline
$==\sum_{i=0}^{k-1}\binom{k}{i}+\sum_{i=0}^{k-1}\binom{k}{k}+ \binom{k}{0}+\sum_{i=1}^{k}\binom{k}{i}$\newline
$==\sum_{i=0}^{k}\binom{k}{i}+\sum_{i=0}^{k}\binom{k}{i}$
$=2^{k}+2^{k}$
$=2*2^{k}=2^{k+1}$\newline
So $  P(k) \rightarrow P(k+1) \square$

}
%%%%%%%%%%%%%%%%%%%%%%%%%%%%%
%Question #3: Part b
%%%%%%%%%%%%%%%%%%%%%%%%%%%%%
\subsection*{Part b}

If $n\geq 1$ then $n2^{n-1}=\sum_{i=0}^{n}i\binom{n}{i}$\\

{\noindent\bf Answer:}
{
Proof:\newline
We used strong induction to prove $2^{n}=\sum_{i=0}^{n}\binom{n}{i}$ \newline
Using that we see that: \newline
$\sum_{i=0}^{n}i\binom{n}{i}=\sum_{i=1}^{n}i\binom{n}{i}$, as $0\binom{n}{0}=0$\newline
$=\sum_{i=1}^{n}n\binom{n-1}{i-1}$ by factoring binomial coefficients\newline
$=\sum_{i=0}^{n-1}n\binom{n-1}{i}$ by taking out 1 index\newline
$=n2^{n-1}$ \newline

}
%%%%%%%%%%%%%%%%%%%%%%%%%%%%%
%Question #3: Part c
%%%%%%%%%%%%%%%%%%%%%%%%%%%%%
\subsection*{Part c}

$\sum_{i=0}^{n}\binom{i}{k}=\binom{n+1}{k+1}$\\

{\noindent\bf Answer:}
{
Proof:\newline
Since we used strong induction to prove binomial coefficients in part a, we can use properties of binomial coefficients to prove this statement\newline
Observe that:\newline
$\sum_{i=0}^{n}\binom{i}{k}=\sum_{i=0}^{n}\binom{k+i}{k}$\newline
$=\sum_{i=-k}^{0}\binom{k+i}{k}+\sum_{i=0}^{n-k}\binom{k+i}{k}$\newline
$=0+\sum_{i=0}^{n-k}\binom{k+i}{k} =\binom{k+n-k+1}{k+1}$\newline
$=\binom{0+n+1}{k+1}=\binom{n+1}{k+1} \square $\newline
}

\newpage
%%%%%%%%%%%%%%%%%%%%%%%%%%%%%%%%%%%%%%%%%%%%%%%%%%%%%%%%%%%%%%%%%%%%
%Question #4
%%%%%%%%%%%%%%%%%%%%%%%%%%%%%%%%%%%%%%%%%%%%%%%%%%%%%%%%%%%%%%%%%%%%
\section*{Question 4}

Let $S\subseteq\mathbb{R}$ s.t. $S$ contains an open interval, i.e. $\exists a,b\in\mathbb{R}$ s.t. $a<b$ and $(a,b):=\{x\in\mathbb{R}:a<x<b\}\subseteq S$. Prove that $\vert S\vert=\vert\mathbb{R}\vert$.\\

{\noindent\bf Answer:}
{
Proof: First, notice that the open interval $(-n\pi,n\pi)$ has the same cardinality as real number line\newline
Lets counstruct a bijection $f:(-n\pi,n\pi) \rightarrow R$, so let $f(x)=sin(x)$\newline
In question 1 we proved that if a function has an inverse then it is bijective\newline
$f^{-1}(x)=arcsin(x)$ so $f$ must be bijective\newline
Now I have to show that $(a,b)$ and  $(-n\pi,n\pi)$ have the same cardinality\newline
Define $g:(a,b)$$\rightarrow (-n \pi,n\pi)$ by letting $g(x)=\pi x$ we see that when $a<x<b$ the function produces outputs in $a\pi<x<b\pi$ which is essentially in  $(-n\pi,n\pi)$ \newline
To show that $g$ is bijective, we have to find its inverse\newline
$g^{-1}(x)=x/\pi$ by inverse formula\newline
$g(g^{-1}(x))=g(x/(\pi))=(\pi*x)/(\pi)=x$, we showed inverse exists\newline
The funtion $g$ has an inverse thus it is bijective(proved in question 1)\newline
That means $(a,b)$ and  $(n\pi,n\pi)$ have the same cardinality,and  $(-n\pi,n\pi)$ has the same cardinality as real\
 number line, by transitivity(proved in question 2) $(a,b)$ have the same cardinality as $R$\newline
Thus $\vert S\vert=\\vert\mathbb{R}\vert$ $\square$

}


\newpage
%%%%%%%%%%%%%%%%%%%%%%%%%%%%%%%%%%%%%%%%%%%%%%%%%%%%%%%%%%%%%%%%%%%%
%Question #5
%%%%%%%%%%%%%%%%%%%%%%%%%%%%%%%%%%%%%%%%%%%%%%%%%%%%%%%%%%%%%%%%%%%%
\section*{Question 5}

NOTE: THIS PROBLEM IS CHALLENGING!\\

{\noindent\it Recall:} The Pigeonhole Principle: If $k,n\in\mathbb{N}$ and we place more than
kn objects into n boxes, then one of these boxes MUST have more than k elements.

\begin{claim}
Suppose we pick n+1 numbers from the set $\{1,...,2n\}$. Prove that the set of n+1 numbers contains a pair of numbers s.t. one is a multiple of the other.
\end{claim}
\noindent (HINT: Write each number in the form $2^{k}m$ with $m\in\mathbb{O}$. Study occurences of values of m using the Pigeonhole Principle.)\\

{\noindent\bf Answer:}
{
Let ${p_{1},p_{2},...p_{n}}$ be integers from set  $\{1,...,2n\}$.\newline
Let $p_{i}=2^{k}m=2^{k}(2m_{i}+1)$ where $1<k<n+1$ and $1<i<n+1$, observe that $2m+1 \in\mathbb{O}$\newline
Since elements $ p_{i}$ are from set $\{1,...,2n\}$, then for $2m+1 \in O$ we can write   $2m+1$ as set $\{1,3,5,..2n-1\}$\newline
By pigeonhole principle  $\exists i, j,i \neq j$ where $ 1<i<n+1$ and $1<j<n+1$ s.t $2m_{i}+1=2m_{j}+1$\newline
By division rule, $p_{i}\mid p_{j}$ and $p_{j}\mid p_{i}$\newline
Thus one of them must be multiple of another.$\square$


}




\end{document}
