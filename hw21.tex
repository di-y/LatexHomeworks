%%%%%%%%%%%%%%%%%%%%%%%%%%%%%%%%%%%%%%%%%%%%%%%%%%%%%%%%%%%%%%%%%%%%
%%%%%%%%%%%%%%%%%%%%%%%%%%%%%%%%%%%%%%%%%%%%%%%%%%%%%%%%%%%%%%%%%%%%
%
%DOCUMENT SETTINGS
%
%%%%%%%%%%%%%%%%%%%%%%%%%%%%%%%%%%%%%%%%%%%%%%%%%%%%%%%%%%%%%%%%%%%%
%%%%%%%%%%%%%%%%%%%%%%%%%%%%%%%%%%%%%%%%%%%%%%%%%%%%%%%%%%%%%%%%%%%%

\documentclass[12pt]{article}

\usepackage{amsmath,amssymb,amsthm,epsfig}

\evensidemargin 0in
\oddsidemargin 0in
\topmargin -.3in
\setlength{\textheight}{8.5in}
\setlength{\textwidth}{6.5in}

\newcommand{\ds}{\displaystyle}
\newcommand{\ul}{\underline}
\newcommand{\vs}{\vspace{3mm}}
\newcommand{\cF}{{\mathcal F}}

\newcounter{inner}

\newcommand{\op}[1]{\operatorname{#1}}
\newcommand{\bx}[1]{\makebox(8,5.5)[c]{#1}}

\newtheorem*{theorem}{Theorem}
\newtheorem*{claim}{Claim}
\begin{document}
%%%%%%%%%%%%%%%%%%%%%%%%%%%%%%%%%%%%%%%%%%%%%%%%%%%%%%%%%%%%%%%%%%%%
%%%%%%%%%%%%%%%%%%%%%%%%%%%%%%%%%%%%%%%%%%%%%%%%%%%%%%%%%%%%%%%%%%%%
%
%HEADER
%
%%%%%%%%%%%%%%%%%%%%%%%%%%%%%%%%%%%%%%%%%%%%%%%%%%%%%%%%%%%%%%%%%%%%
%%%%%%%%%%%%%%%%%%%%%%%%%%%%%%%%%%%%%%%%%%%%%%%%%%%%%%%%%%%%%%%%%%%%
\begin{center}
\hrule
\vskip .2in
\centerline{\bf \Large 21-127-T: Concepts of Mathematics}
\centerline{\bf Homework 2}
{\bf Due date: 5/30/2015, 11:00 PM}
\vskip .2in
\hrule
\end{center}
\thispagestyle{empty}
{\bf \noindent Name:Dina Yerlan \newline Collaborators:0}
\vspace {0.2in}
\hrule
\vspace {0.2in}

%%%%%%%%%%%%%%%%%%%%%%%%%%%%%%%%%%%%%%%%%%%%%%%%%%%%%%%%%%%%%%%%%%%%
%%%%%%%%%%%%%%%%%%%%%%%%%%%%%%%%%%%%%%%%%%%%%%%%%%%%%%%%%%%%%%%%%%%%
%
%HOMEWORK BODY
%
%%%%%%%%%%%%%%%%%%%%%%%%%%%%%%%%%%%%%%%%%%%%%%%%%%%%%%%%%%%%%%%%%%%%
%%%%%%%%%%%%%%%%%%%%%%%%%%%%%%%%%%%%%%%%%%%%%%%%%%%%%%%%%%%%%%%%%%%%

%%%%%%%%%%%%%%%%%%%%%%%%%%%%%%%%%%%%%%%%%%%%%%%%%%%%%%%%%%%%%%%%%%%%
%Question #1
%%%%%%%%%%%%%%%%%%%%%%%%%%%%%%%%%%%%%%%%%%%%%%%%%%%%%%%%%%%%%%%%%%%%
\section*{Question 1}

Do problem 4, parts (a)-(c) from section 1.6. Note that the statement $v$ is easier to understand if written as
\[
v\equiv\{\exists M,K\in\mathbb{R}\text{ s.t. }\forall x\in\mathbb{R}, f(x)>M\}
\]
%%%%%%%%%%%%%%%%%%%%%%%%%%%%%
%Question #1: Part a
%%%%%%%%%%%%%%%%%%%%%%%%%%%%%
\subsection*{Part a - Answer:}

%%%%%%%%%%%%%%%%%%%%%%%%%%%%%
%Question #1: Part b
%%%%%%%%%%%%%%%%%%%%%%%%%%%%%
\subsection*{Part b - Answer:}

%%%%%%%%%%%%%%%%%%%%%%%%%%%%%
%Question #1: Part c
%%%%%%%%%%%%%%%%%%%%%%%%%%%%%
\subsection*{Part c - Answer:}


\newpage
%%%%%%%%%%%%%%%%%%%%%%%%%%%%%%%%%%%%%%%%%%%%%%%%%%%%%%%%%%%%%%%%%%%%
%Question #2
%%%%%%%%%%%%%%%%%%%%%%%%%%%%%%%%%%%%%%%%%%%%%%%%%%%%%%%%%%%%%%%%%%%%
\section*{Question 2}

Do problem 2, parts (d)-(f) from section 1.6. Note that the statement $v$ is easier to understand if written as
\[
v\equiv\{\exists M,K\in\mathbb{R}\text{ s.t. }\forall x\in\mathbb{R}, f(x)>M\}
\]
%%%%%%%%%%%%%%%%%%%%%%%%%%%%%
%Question #2: Part d
%%%%%%%%%%%%%%%%%%%%%%%%%%%%%
\subsection*{Part d - Answer:}

%%%%%%%%%%%%%%%%%%%%%%%%%%%%%
%Question #2: Part e
%%%%%%%%%%%%%%%%%%%%%%%%%%%%%
\subsection*{Part e - Answer:}

%%%%%%%%%%%%%%%%%%%%%%%%%%%%%
%Question #2: Part f
%%%%%%%%%%%%%%%%%%%%%%%%%%%%%
\subsection*{Part f - Answer:}

\newpage
%%%%%%%%%%%%%%%%%%%%%%%%%%%%%%%%%%%%%%%%%%%%%%%%%%%%%%%%%%%%%%%%%%%%
%Question #3
%%%%%%%%%%%%%%%%%%%%%%%%%%%%%%%%%%%%%%%%%%%%%%%%%%%%%%%%%%%%%%%%%%%%
\section*{Question 3}

Prove the following claim:
\begin{claim}
There exists no largest real number.
\end{claim}

{\noindent\bf Answer:}
{
 \newline
 \newline
Claim: There exists no largest real number\newline
 \newline
Given(negation of the claim):  $\{\exists M\in R\text{ s.t. } \forall x\in R\ M>x \}$\newline
 \newline
Proof: By contradiction assume there exists the largest real number\newline
 \newline
Choose $M_{0}\in R$ s.t $\forall x\in R$ $M_{0}>x$\newline
 \newline
Let $x=M_{0}+1$\newline
 \newline
Then, $M_{0}>M_{0}+1$\newline
 \newline
Witness that $M_{0}>M_{0}+1$ cannot be true\newline
 \newline
Thus our assumption was wrong\newline
}


\newpage
%%%%%%%%%%%%%%%%%%%%%%%%%%%%%%%%%%%%%%%%%%%%%%%%%%%%%%%%%%%%%%%%%%%%
%Question #4
%%%%%%%%%%%%%%%%%%%%%%%%%%%%%%%%%%%%%%%%%%%%%%%%%%%%%%%%%%%%%%%%%%%%
\section*{Question 4}

Suppose $D$ is some set and $P(x)$ is a predicate function. Mathematicians will often write claims of the form:\\

{\it\noindent There exists a unique (meaning there's one and ONLY one) element of a given set, $D$, satisfying a given property, $P(x)$.}\\

\noindent To denote the phrase ``there exists a unique element'' we use - the fake, but still useful for shorthand - quantifier ``$\exists !$''. For example, if D is a set and P(x) is a predicate function we would write the statement ``there exists a unique element of D s.t. P(x) is true'' as
\[
\{\exists ! x\in D\text{ s.t. }P(x)\}
\]
To prove the statement above we {\it usually} proceed as follows:
\begin{enumerate}
  \item Find/build a specific $x_{0}\in D$ s.t. $P(x_{0})$ is true, i.e. we prove the statement
    \[
    \exists x\in D\text{ s.t. }P(x)
    \]
    Remember that once we find/build a candidate $x_{0}$ we must prove that $P(x_{0})$ is true.
    \item We prove that if an element in D satisfies the property $P(x)$ then it's unique. To do so, we often proceed by contradiction and occasionally will use the $x_{0}$ we defined in the first step.
\end{enumerate}
%%%%%%%%%%%%%%%%%%%%%%%%%%%%%
%Question #4: Part a
%%%%%%%%%%%%%%%%%%%%%%%%%%%%%
\subsection*{Part a}
Let $D$ be a set, $P(x)$ be a predicate function, and $x_{0}\in D$ s.t. $P(x_{0})$ is true. Write what it means for $x_{0}$ to be the unique element of $D$ satisfying $P(x)$. Also, write the negation of this statement. (HINT: The non-negated statement will involve the quantifier $\forall$).\newline

{\noindent
Claim: $\{\exists ! x_{0}\in D\text{ s.t. }P(x_{0})\}$\newline
. \newline
Meaning: It means that there is one and only one element in D,$x_{0}$ that makes the expression $P(x_{0})$ true\newline
. \newline
Negation: $\{\forall  x_{0}\in D,not P(x_{0})\}$\newline
. \newline
}
%%%%%%%%%%%%%%%%%%%%%%%%%%%%%
%Question #4: Part b
%%%%%%%%%%%%%%%%%%%%%%%%%%%%%
\subsection*{Part b}

Using your answer above, write the statement $\{\exists ! x\in D\text{ s.t. }P(x)\}$ using the quantifiers $\forall$ and $\exists$.\newline

{\noindent\bf Answer:}
{
$(\exists x,P(x)) \land (\forall y,P(y))$ $\Rightarrow$ $(x=y)$
}
%%%%%%%%%%%%%%%%%%%%%%%%%%%%%
%Question #4: Part c
%%%%%%%%%%%%%%%%%%%%%%%%%%%%%
\subsection*{Part c}

Prove the following claim
\begin{claim}
If n is an even integer there exists a unique $i\in\mathbb{Z}$ s.t. $n=2i$.
\end{claim}

{\noindent\bf Answer:}

\newpage
%%%%%%%%%%%%%%%%%%%%%%%%%%%%%%%%%%%%%%%%%%%%%%%%%%%%%%%%%%%%%%%%%%%%
%Question #5
%%%%%%%%%%%%%%%%%%%%%%%%%%%%%%%%%%%%%%%%%%%%%%%%%%%%%%%%%%%%%%%%%%%%
\section*{Question 5}

Prove the following claim by contraposition:
\begin{claim}
Let $n,m,p\in\mathbb{Z}$. If $nmp$ is even then $n$, $m$, or $p$ is even.
\end{claim}

{\noindent\bf Answer:}
{\noindent
Claim(Using quantifiers): $n,m,p\in\mathbb{Z}$ $\{nmp\in E\}$ $\Rightarrow$ $\{n \in E \lor m \in E \lor p \in E\}$ .\newline
.\newline
Contraposition: $n,m,p\in\mathbb{Z}$  $\{n \in O \land m \in O \land p \in O\}$ $\Rightarrow$  $\{nmp\in O\}$ \newline
.\newline
NTS:  $\exists k \in Z$ s.t $n_{0}m_{0}p_{0}=2k+1 $\newline
.\newline
Proof: By contraposition. Choose $n_{0},m_{0},p_{0} \in Z $ \newline
.\newline
$\exists a,b,c \in Z$ where $n_{0}=2a+1$, $m_{0}=2b+1$, $p_{0}=2c+1$\newline
.\newline
Observe that  $n_{0}m_{0}p_{0}$=$(2a+1)(2b+1)(2c+1)$ \newline
.\newline
              $=$ $(4ab+2a+2b+1)(2b+1) $ $=$\newline
.\newline
              $=$ $8ab^{2}+8ab+4b^{2}+4b+2a+1$ $=$\newline
.\newline
              $=$ $2(4ab^{2}+4ab+2b^{2}+2b+a)+1$\newline
.\newline
Let $k=4ab^{2}+4ab+2b^{2}+2b+a $ then $n_{0}m_{0}p_{0}=2k+1$\newline
.\newline
We see that $n_{0}m_{0}p_{0}$ is odd, so we proved that $\{n_{0}m_{0}p_{0}\in O\}$\newline
.\newline
}
\end{document}
